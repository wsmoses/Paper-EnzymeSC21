\documentclass{standalone}
\usepackage{pgfplots}
\usepackage{pgfkeys}
\usepackage{ifthen}
\usepackage{pgfplotstable}
\usetikzlibrary{arrows,positioning}
\usepgfplotslibrary{colorbrewer}
\pgfplotsset{compat=1.16}

\pgfplotstableread[col sep=comma]{
AD, CacheLICM, Inlining, PreOpt
4.7,   9.5,  6372.2, 15000
}\RSBench

\pgfplotstableread[col sep=comma]{
% AD, Templating, PHI, AliasAnalysis, Recompute, PreOptimization
% 3.2, 9.6, 14.5, 47.4, 95.8, 15000
AD, Templating, PHI,LoopBound, PreOptimization
3.2, 9.5, 16.3, 25.9, 15000
}\XSBench
% 3.16, 9.61, 14.47, 47.40, 95.75, 15000

\pgfplotstableread[col sep=comma]{
AD, Allocator, Recompute, InlineCacheABI
6.4, 8.7, 19.87, 15000
}\LBM
%6.39, 8.74, 19.87, 15000


\pgfplotstableread[col sep=comma]{
AD, SpecPHI, PreOptimization
2.0, 2.4, 2979.1
}\LULESH

\pgfplotstableread[col sep=comma]{
AD, Unrolling, MallocCoalescing, PreOptimization
17.8, 116.6, 1378.3, 15000
}\DGCUDA
% 17.85, 116.63, 1378.32, 15000

\pgfplotstableread[col sep=comma]{
AD, Unrolling
5.4, 15000
}\DGROCM

\pgfplotsset{
    every axis/.style={
        % xbar,
        xbar stacked,
        xmode = log,
        point meta=rawx,
        bar shift auto=0pt,
        % bar width={1/2},
        xlabel={Overhead above Forward Pass},
        bar width={1/8},
        ymin=0.9,ymax=7,
        xmin=1,xmax=15000,
        ytick={1,2,3,4,5,6,7},
        yticklabels={XSBench, RSBench, LULESH, LBM, DG (CUDA), DG (ROCm)},
        xtick={1,10,100,1000,14000},
        xticklabels={Forward (1x),10x,100x,1000x,OOM},
        height=7cm, scale only axis,
        width=22cm,
        y tick label as interval,
        nodes near coords align={horizontal},
        ymajorgrids
    },
        /pgfplots/bar cycle list/.style={/pgfplots/cycle list={
            {Set1-A,fill=Set1-A!50!white,mark=none},
            {Set1-B,fill=Set1-B!50!white,mark=none},
            {Set1-C,fill=Set1-C!50!white,mark=none},
            {Set1-D,fill=Set1-D!50!white,mark=none},
            {Set1-E,fill=Set1-E!50!white,mark=none},
            {Set1-F,fill=Set1-F!50!white,mark=none},
            {Set1-G,fill=Set1-G!50!white,mark=none},
            {Set1-H,fill=Set1-H!50!white,mark=none},
            },
            },
}

\begin{document}%
\begin{tikzpicture}

\def\x#1#2#3#4{%
  \begingroup\edef\z{\endgroup\noexpand#1{#2}{#3}{#4}}\z
}

\newcommand{\makenode}[3]{%

 \ifthenelse{\equal{\detokenize{#1}}{\detokenize{15000}}}{
     \node[inner sep=-2,outer sep=0] (#3) at (axis cs:#1*.92, #2) {$\mathbin{\tikz [x=3.4ex,y=3.4ex,line width=1.2ex, red] \draw (0,0) -- (1,1) (0,1) -- (1,0);}$};
}{
 \node[inner sep=0,outer sep=0] (#3) at (axis cs:#1, #2) {o};
 
\ifthenelse{\equal{\detokenize{#1}}{2.0}}{

 \node[inner sep=0,outer sep=0, below left=1em and 0.5em of #3, anchor=center] at (axis cs:#1, #2) {\small $#1\times$};
 
}{
 \ifthenelse{\equal{\detokenize{#1}}{2.4}}{
 
 \node[inner sep=0,outer sep=0, below right=1em and 0.7em of #3, anchor=center] at (axis cs:#1, #2) {\small $#1\times$};
 
}{

 \node[inner sep=0,outer sep=0, below=1em of #3, anchor=center] at (axis cs:#1, #2) {\small $#1\times$};
 
}
}

 }
}

\newcommand{\makearrow}[3]{%
\if{#2}{AD}
\else

 \ifthenelse{\equal{\detokenize{#3}}{\detokenize{Allocator}}}{
\draw [-stealth, ultra thick](#1) -- node [above left=0.9em and 0.6em,midway, anchor=center] {#3} (#2);
}{
\ifthenelse{\equal{\detokenize{#3}}{\detokenize{Recompute}}}{
\draw [-stealth, ultra thick](#1) -- node [above right=0.8em and 0.6em,midway, anchor=center] {#3} (#2);
}{
 \ifthenelse{\equal{\detokenize{#3}}{\detokenize{PHI}}}{
\draw [-stealth, ultra thick](#1) -- node [above left=0.9em and 0.3em,midway, anchor=center] {#3} (#2);
}{
 \ifthenelse{\equal{\detokenize{#3}}{\detokenize{LoopBound}}}{
\draw [-stealth, ultra thick](#1) -- node [above right=0.8em and 0.4em,midway, anchor=center] {#3} (#2);
}{
\draw [-stealth, ultra thick](#1) -- node [above=0.8em,midway, anchor=center] {#3} (#2);
}
}
}
}
\fi
}

\newcommand{\parsetable}[3]{
\pgfplotstableforeachcolumn#1\as\col{
\pgfplotstableforeachcolumnelement{\col}\of#1\as\cell{%
\x\makenode{\cell}{0.5+#3}{#2\pgfplotstablecol}
% \x\makenode{\cell}{1/7+\pgfplotstablecol/7+#3}{#2\pgfplotstablecol}
}
}

% \x\makearrow{0#2}{1#2}{}
}

\begin{axis}

 
\parsetable{\XSBench}{XSBench}{1}
\parsetable{\RSBench}{RSBench}{2}
\parsetable{\LULESH}{LULESH}{3}
\parsetable{\LBM}{LBM}{4}
\parsetable{\DGCUDA}{DGCUDA}{5}
\parsetable{\DGROCM}{DGROCM}{6}

\newcommand{\forward}[1]{
% \node[inner sep=0,outer sep=0] (FW#1) at (axis cs:1, 0.45+#1) {o};
% \node[inner sep=0,outer sep=0] at (axis cs:1, 0.7+#1) {\small Forward};
}
\forward{1}
\forward{2}
\forward{3}
\forward{4}
\forward{5}
\forward{6}
% \parsetable{\RSBench}{RSBench}{5}

\end{axis}

\newcommand{\sub}[1]{\pgfmathparse{int(#1-1)}\pgfmathresult}


\newcommand{\conditional}[2]{%
\ifthenelse{\equal{\detokenize{#1}}{\detokenize{AD}}}{}{#2}
}

\newcommand{\makearrows}[2]{
\pgfplotstableforeachcolumn#1\as\col{
\pgfplotstableforeachcolumnelement{\col}\of#1\as\cell{%
\expandafter\conditional\expandafter{\col}{
\pgfmathparse{\pgfplotstablecol-1}
\x\makearrow{#2\pgfplotstablecol}{#2\pgfmathresult}{\col}
}
}
}
}

\makearrows{\DGROCM}{DGROCM}
\makearrows{\DGCUDA}{DGCUDA}
\makearrows{\LBM}{LBM}
\makearrows{\LULESH}{LULESH}
\makearrows{\RSBench}{RSBench}
\makearrows{\XSBench}{XSBench}




\end{tikzpicture}
\end{document}

% % and look like this
% \begin{tikzpicture}
% \foreach \row in {0,1} {
%   \pgfplotstablegetelem{\row}{x}\of\loadedtable
%   \let\x=\pgfplotsretval
%   \pgfplotstablegetelem{\row}{y}\of\loadedtable
%   \let\y=\pgfplotsretval
%   \draw (0,0) -- (\x,\y);
% }
% \end{tikzpicture}
% \end{document}